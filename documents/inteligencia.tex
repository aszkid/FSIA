\section{Què és?}

La intel·ligència artificial és aquella branca de la informàtica dedicada al
desenvolupament d'algorismes per a conseguir que una màquina prengui decissions
 racionals per si mateixa o que es comporti de forma similar a la intel·ligència humana.

En resumides comptes, segons la definició més estesa, que és la de l'informàtic i
investigador cognitiu estadunidenc \emph{John McCarthy}, és \emph{"Fer que una màquina
es comporti d'una manera que en un humà considerariem intel·ligent"}.

En el que a robòtica es refereix la intel·ligència artificial consisteix a aplicar la
definició anterior, és a dir, un ésser no viu amb una intel·ligència racional semblant
a la humana, a una estructura que sol tenir una fisiologia semblant a la nostre i que
es pot moure.
\cite{definiciondeia} \cite{wikiia} \cite{monoia}

\section{Principals utilitats}

La IA s'utilitza diàriament en àmbits extremadament diversos: diagnosi mèdica, comerç amb stock, control de robots, llei, traducció de textos, domòtica, etc.




