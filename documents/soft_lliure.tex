\section{Què és?}

El software lliure (de l'anglès \emph{"free software"}) és tot aquell software publicat
sota llicències que respecten el concepte de \emph{llibertat}. Degut a que la definició
(en relació al software) de llibertat és molt àmplia, i es podria dedicar inacabable espai
a definir-la, ho farem simple; software és lliure si al adquirir-lo, l'usuari pot fer-lo servir,
copiar-lo, estudiar-lo, modificar-lo, i redistribuir-lo lliurement de diferents formes. \cite{wikifree}

Que el software sigui lliure, no implica que el seu cost sigui zero (encara que molt
software lliure, també és gratuït). Per tant, jo puc crear software lliure i distribuir-lo a
\EUR{500,000} la còpia; una altra cosa es que n'aconsegueixi vendre alguna.
Per això, qui \emph{ven} software lliure, ho fa a un preu raonable. El desenvolupament de software
lliure també es financia a través de donacions voluntàries, i la major part de projectes dediquen
aquests diners recaudats al manteniment dels serveis que el software que distribueixen ofereix.

\section{Qui el fa?}

No hi han companyies multimilionàries que facin software lliure (encara que n'hi ha bastantes que
col·laboren al seu desenvolupament). Una bona forma de analitzar la quantitat de software lliure
que hi ha, es fer una búsqueda a internet de projectes llicenciats sota llicències denominades
\emph{lliures} per la \emph{\ac{fsf}} \cite{licenses}

\section{Història}

\section{Us actual}

\section{Avantatges}
