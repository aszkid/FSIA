Les \emph{xarxes neuronals artificials} (ANN) són \emph{models computacionals 
inspirats pel sistema nerviós central dels animals (en particular, el cervell)} \autocite{nnpatrec}.

L'estructura d'una \ac{ann} és semblant a la d'un circuit electrònic: hi ha una capa o \emph{layer}
de $ q $ neurones d'entrada, o \emph{inputs}, i un layer de $ p $ neurones de sortida, o \emph{outputs}.
Això permet representar la seva estructura com si fos una funció matemàtica, $ f(x_0, x_1, ..., x_q) = \{y_0, y_1, ..., y_p\} $,
és a dir: un conjunt $ x $ d'entrades formen un conjunt $ y $ de sortides \ref{simple_ann}.

\begin{figure}[ht!]
\centering
\includegraphics[width=80mm]{data/nn_simple.png}
\caption{El propòsit del layer \emph{hidden} s'explicarà més endavant.}
\label{simple_ann}
\end{figure}

La tasca principal d'una xarxa neuronal \emph{supervisada} és realitzar una \emph{aproximació de funció},
és a dir, a partir d'un seguit de valors de la funció (anomenat \emph{training set}), $ f(a) = b $, crear un model que s'ajusti a les dades
que se li ha subministrat \ref{func_approx}. Podríem, per exemple, entrenar una \ac{ann} per a que realitzés
una aproximació de la funció $ sin(x)$, però és més interessant entrenar-la amb objectius més sofisticats; tot
problema determinista que es pugui reduir a \emph{produir unes sortides a partir d'uns valors d'entrada}, es pot
solucionar, molt probablement, amb una xarxa neuronal.

\begin{itemize}
\item \emph{Conduir un cotxe:} la posició, la velocitat, els cotxes adjacents, la carretera... una gran quantitat d'entrades. La nova direcció, 
accelerar o frenar, com a sortides.
\item \emph{Predir el flux del mercat de valors:} la pendent de la gràfica del valor d'una acció, els últims moviments, els últims compradors...
la xarxa s'ha entrenat amb dades històriques, i l'historia es repeteix. El moviment de l'accionista com a sortida.
\item \emph{Reconèixer caràcters:} aquesta és la nostra parada per a continuar amb l'explicació de les \ac{ann}. El conjunt de píxels que formen 
l'imatge d'un caràcter com a entrades, i el número que representen els píxels com a sortida. És un exemple fàcil d'aplicar i enriquidor.
\end{itemize}

\begin{figure}[ht!]
\centering
\includegraphics[width=80mm]{data/func_approx.jpg}
\caption{Exemple d'aproximació de funció. Els punts verds són els valors subministrats, el training set,
i la línia blava la funció que s'ha aproximat.}
\label{func_approx}
\end{figure}