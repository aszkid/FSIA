Els procediments del projecte de recerca es basen en la realització de diferents \emph{demos} (demostracions)
que tenen la intenció de oferir una introducció a les aplicacions de la \ac{ia}.

S'ha utilitzat el llenguatge de programació C++, i les llibreríes SFML \autocite{sfmllib} (per a gràfics) i FANN \cite{fannlib} (per a la
creació i entrenament de xarxes neuronals).

Les demos (junt amb el seu estat de progrés actual) són les següents:

\begin{itemize}
\item \emph{Demo 1} - joc de futbol sala (5 jugadors per equip), aprenentatje 'per esforç' (reinforcement) amb \ac{ann} evolutiva, estat inicial
\item \emph{Demo 2} - reconeixement de caràcters 0-9, \ac{ann}, finalitzat +- (ajustament de paràmetres d'aprenentatge, etc)
\item \emph{Demo 3} - conducció automàtica (circuit simple), reinforcement amb \ac{ann} evolutiva, estat inicial (físiques de conducció)
\item \emph{Demo 4} - joc de taula (similar a \href{http://gabrielecirulli.github.io/2048/}{això}), no iniciat
\end{itemize}