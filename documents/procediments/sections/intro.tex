Els procediments del projecte de recerca es basen en la realització de diferents \emph{demos} (demostracions)
que tenen la intenció de oferir una introducció a les aplicacions de la \ac{ia}.

S'ha utilitzat el llenguatge de programació C++, i les llibreríes SFML \autocite{sfmllib} (per a gràfics) i FANN \cite{fannlib} (per a la
creació i entrenament de xarxes neuronals).

Les demos que hem realitzat són les següents:

\begin{itemize}
\item Reconeixement de caràcters 0-9, a través de l'ús d'una xarxa neuronal.
\item Conducció automàtica (circuit simple), reinforcement learning a través de l'algoritme \emph{Q-learning}.
\end{itemize}
