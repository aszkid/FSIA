Després d'aquest llarg camí, hem de respondre a la pregunta que ens vam fer a principi de curs:
\emph{és possible crear intel·ligència artificial?}

Podem respondre a la pregunta de dues formes completament diferents i oposades, tot depenent
de la magnitud implícita amb que es faci la pregunta:
\begin{enumerate}
\item \emph{Màquines pensants, que resolen problemes, parlen, comprenen...}: No. Ni nosaltres,
amb els nostres escassos recursos i coneixements, ni les majors empreses del món, amb quasi tots
els recursos i coneixements, som capaços de crear autòmats que simulin l'ésser humà en tasques
cognitives \footnote{Acció de pensar, raonar} i motrius. Encara estem molt lluny d'arribar a això, però...
\item \emph{Aplicacions a petita escala, que solucionen problemes específics}: Sí! Nosaltres,
amb coneixements rudimentaris sobre la matèria, hem pogut solucionar dos problemes típics d'éssers humans:
llegir caràcters i conduir un automòbil. Grans empreses, son capaces de solucionar aquests problemes amb
molta més profunditat (llegir paraules senceres, ana- litzar oracions, conduir automòbils en un ambient hostil, etc).
\end{enumerate}

Per tant, veiem que és qüestió de \emph{generalitzar} processos; ser capaços de, amb un sol algoritme,
aprendre a conduir i llegir paraules i oracions, caminar, conduir, saltar, raonar, emmagatzemar, classificar,
i processar idees i informació... una tasca realment titànica. Hi ha projectes lliures, com \href{http://opencog.org/}{OpenCog}
, que
pretenen crear això: un programa o conjunt d'utilitats que, a través d'una interfície sòlida i modular, sigui
capaç de simular un cervell humà.

Ens hem trobat amb infinitat de problemes durant la recerca; papers científics altament complexos,
textos de divulgació no suficientment profunts, problemes relacionats amb la programació i implementació
dels algoritmes que hem utilitzat, i problemes també en intentar mantenir un format i estil de treball
conjunt. Però, ha sigut una experiència molt gratificant, que ens ha permès ampliar els nostres coneixements
i àrees d'interès.
