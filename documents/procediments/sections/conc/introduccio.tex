Aquest projecte de recerca es divideix en dues parts; els \emph{conceptes}
i els \emph{procediments}.

El \emph{software}, les seves filosofies i llicències formaran el primer i més
teòric apartat. S'explicarà què és el software i es separarà en \emph{propietari}
i \emph{lliure}, dues formes de veure la creació i distribució del mateix. A més,
s'introduirà la \emph{intel·ligència artificial} (IA), la seva història, el seu present i el
seu hipotètic futur, com a preludi pels procediments del projecte.

En la segona part, sobre \emph{intel·ligència artificial}, s'explicarà el procés
que s'ha seguit per a crear un seguit de \emph{demostracions}, que apliquen un
dels mètodes més coneguts de la IA; les \emph{xarxes neuronals}. A través de demostracions
gràfiques es podrà entendre millor en concepte de les ANN (\emph{Artificial Neural Networks}).

Amb aquest projecte es pretén trobar un punt en comú entre les matemàtiques,
la intel·ligència artificial i el software (específicament, lliure), a més de
respondre a una pregunta que molts ens haurem fet en algun moment: \emph{és possible crear
intel·ligència artificial?}

S'ha treballat de forma altament col·laborativa, a través d'una plataforma
en línia anomenada \href{http://github.com}{GitHub}, que implementant un sistema
de control de versions anomenat \href{http://git-scm.com/}{Git}, descentralitza
el projecte, i permet el lliure accés a col·laboradors externs i, especialment, a membres
del mateix grup. Permet mantenir en sincronització el projecte i fusionar canvis entre els integrants del grup.

Per a realitzar l'escrit, s'ha utilitzat \LaTeX, un llenguatge de programació que permet escriure documents de forma molt més flexible, multiplataforma i lleugera, a més de facilitar la col·laboració entre diferents persones. També introdueix una forma d'escriure la bibliografia molt més pragmàtica i viable, que actualitza les referències sense haver de fer-ho manualment, i crea l'índex automàticament.
