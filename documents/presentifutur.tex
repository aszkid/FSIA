\section Introducció
La funció de l'intel·ligència artificial és fer accions que es consideren intel·ligents. Aquestes accions poden requerir robots, com ara construir la peça de un cotxe,
però d'altres només necessiten un programa informàtic, com per exemple un dels programes més utilitzats alhora de traduir textos a altres llengües: el google traductor.
Per tant la intel·ligència artificial intenta fer accions humanes de una manera intel·ligent i ,fins i tot, més eficient amb l'objectiu de millorar 
facilitar la vida a la raça humana.
Això desperta varies preguntes: Què es pot fer actualment amb l'intel·ligència artificial? Té futur o és simplement un projecte que porta a un carreró sense sortida? Pot
ser perillosa i, com en moltes pel·licules de ciència ficció passa, que ens intentin eliminar a la humanitat? Per què no es creen màquines intel·ligents de veritat?
Aquestes preguntes seràn contestades en aquest apartat.

\section Present de la IA
Actualment s'està treballant per crear intel·ligència artificial capaç de fer accions humanes com ara la visió o la manipulació d'objectes. Tot i que semblen accions facils
són molt dificils de implementar en programació perquè actuïn de manera intel·ligent. En el nostre dia a dia i han molts programes que utilitzen intel·ligència artificial:
google traductor, traductor instantani que haviem esmentat en la introducció, té una complexitat que no s'observa a simple vista ja que sembla molt fàcil buscar al diccionari
i buscar les paraules que et demanen, però no es té en compte que aquest programa té que dividir el text en parts i buscar les paraules adïents per a cada un dels casos seguint
les normes ortogràfiques i mantenint el significat del text original. En l'àmbit dels programes de l'intel·ligància artificial sense robòtica també podem incluir les recomenacions
de canals que t'ofereix youtube segons els teus gustos o les recomenacions de productes que moltes pàgines web, com ara ebay o amazon, t'ofereixen. També s'esta treballant creant
programes que et responen de manera llògica i raonada però moltes vegades acaba en fracàs i les seves respostes no tenen cap sentit, com pot ser el cas de cleverbot, o les seves
respostes són simples i pervisibles.
Pel que fà l'àmbit de la robòtica s'han creat robots capaços de caminar de manera òptima i evitar caigudes causats per els diferents tipus de terreny intentant simular animals com
ara gossos alhora de correr. També s'han creat humanoides, robots amb forma de persones, que són capaços de fer moviments humans, per exemple abraçar expressar emocions(no sentir-les),
manipular certs objectes... però fent accions de manera molt específica.
Un últim cas d'intel·ligència artificial és la que utilitzats maquinària externa per fer funcionar un programa infomàtic. Aquest cas és els de reconeixement de veu natural ja que no
fà cap acció que es pugui veure a l'exterior però gràcies al micròfon que tenen fà que siguin capaces de analitzar la teva veu. Apple està experimentant amb aquest últim cas en els
seus nous productes amb un programa anomenat SIRI, un reconeixedor de veu que també és capaç de interpretar-la i respondret a la teva pregunta.

\section Futur de la IA
És difícil predir quin serà el futur de la intel·ligència artificial ja que tot són hipòtesis que probablement no seràn certes, però ens podem fer una idea de quins objectius de futur
pot tenir l'intel·ligència artificial.
Un dels objectius que tenen la majoria de persones que treballen en la IA és aconseguir que aquesta intel·ligència deixi de ser tant espècifica i passi a ser global capaç d'aprendre
qualsevol cosa, analitzar-la de manera intel·ligent i pensar bàsicament com les persones humanes. Aquest objectiu encara és difícil d'aconseguir perquè si volem que pensin com a éssers
humans primer hem de saber perquè pensem com a éssers humans. Aquest concepte pot semblar estrany però el que pretenem dir és que hem d'entendre primer com funciona el nostre cervell
per raonar, com reacciona en situacions que no coneix i quina és la seva reacció davant d'aquestes situacions. Per tant podem dir que la IA pot estar lligada a la biologia i que, en part,
depen d'ella.
Un altre dels objectius de futur de la IA és el que fa referència a la intel·ligencia específica i en aquest cas està més a l'abast dels programadors. Un dels molts projectes de aquesta
intel·ligència és el de cotxes que siguin capaços de conduir sense la necessitat de un conductor. Aquest programa podria evitar molts accidents causats per la consumició de drogues o
d'alcohol fent més segures les carreteres i evitant errors humans que a qualsevol persona podem passar-li. Es podria pensar que llavors la figura de un taxista o de un conductor de
autobus és inutil ja que una màquina faria la seva funció però això és fals ja que no es podria eliminar el factor humà en aquests seveis per el simple fet de que si hi ha un error en la
en el programa informàtic o en la maquinària del vehicle es necessitaria una persona per fer aquesta feina i per tant s'aconseguiria un estat de simbiosi. Aquest exemple és un dels multiples
objectius de la intel·ligència artificial específica i podriem estar-ne dient molts més ja que és un camp molt ampli i que encara es pot explotar.

\section Perillositat de la IA
La discusió de si pot arribar a ser perillosa o no una IA capaç de pensar com un humà dóna molt a parlar i discutir ja que s'han de tenir en compte molts factors i ara direm la nostra
opinió.
Si en algún moment s'aconsegueix una intel·ligència artificial capaç de pensar com els humans però de una manera més eficient i calculadora, si tingués ment per pensar i raonar podria arribar
a pensar que nosaltres, els éssers humans, no tenim cap funció ja que fariem el mateix que fan elles però de una manera menys eficient amb més defectes i això la podria fer pensar que hauriem
de desapareixer i per tant ella voldria eliminar als seus creadors. No seria perquè ens tingués odi, però tampoc ens tindria amor ni apreci, simplement pensaria que som inservibles i per tant
ens eliminaria com si nosaltres tiressim una joguina o un electrodomèstic a la paperera perquè ja l'únic que fa es ocupar-nos espai que podria ser utilitzat per una altre cosa, i pel que fa
els éssers humans estariem gastant recursos que podrien ser utilitzats per recerca. Aquesta situació espantosa i clarament alarmant es podria evitar aplicant un dels dos factors més importants
que fan els ésser humans tal com són: els valors i les emocions. Si ens preguntessin que ens diferencia del animals diriem que clarament és la intel·ligència, que en part és veritat, però el
factors que ens permet viure en comunitat i diferenciar-nos dels animals són els valors, que ens impedeixen fer accions que estan en contra la nostra moral, i les emocions, que ens permeten
reaccionar amb l'exterior. Per tant si una màquina té valors i emocions i nosaltres la tractem com un igual no tindria el pensament de eliminar-nos i podria viure ajudant-nos màquines i humans.

\section Per què una evolució tant "lenta"?
Es podria dir que hem avançat molt en el camp de la intel·ligència artificial els últims anys ha estat rapida i en part és veritat però en realitat no ha sufrit una evolució grandiosa.Això és
donat per diversos factors:

